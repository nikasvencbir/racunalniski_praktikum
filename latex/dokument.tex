\documentclass[10pt]{article}
\usepackage[a4paper, total={6in, 11.5in}]{geometry}
\usepackage[T1]{fontenc}
\usepackage[utf8]{inputenc}
\usepackage[slovene]{babel}
\usepackage{amsmath}
\usepackage{amsthm}
\usepackage{amssymb}
\usepackage{graphicx}


\newcommand{\sinxinv}{}

\begin{document}
\thispagestyle{empty}

% 1. naloga:
% Varšavski lok
% Karol Borsuk
% 20. januar 2026


Topologija je matematična veda, ki proučuje pojem \emph{prostor}. Študenti
matematike v prvem letniku spoznajo ekvlidske, vektorske in metrične prostore. A
topologi poznajo še veliko drugih prostorov, ki imajo nenavadne lastnosti. Eden takih je
tudi \emph{varšavski lok}, ki ga definiramo takole.

% 3. naloga: 
% začetek definicije
  Varšavski lok $L$ je topološki podprostor ravnine, definiran s predpisom
  \begin{equation*}
    L \mathrel{{:}{=}} \left\{ \left(x, \sinxinv \right) \mid x \in (0,1]\} \cup \{(0,0) \right\}.
  \end{equation*}
% konec definicije

% 6. naloga: 
Na sliki~!! vidimo graf funkcije $y = \sinxinv$ na polodprtem intervalu $(0,1]$.
Če grafu dodamo še izhodišče~$(0,0)$, dobimo varšavki lok.
% 
\begin{figure}[ht]
  \centering

  \label{fig:sin-1-x}
\end{figure}
%

Varšavski lok ima naslednje lastnosti:
% 
% 4. naloga: 
% 
% začetek seznama
% lastnost 1
$L$ je povezan prostor, kar pomeni, da ga ne moremo ga razdeliti na disjunktni odprti neprazni množici.
% 
% lastnost 2
$L$ ni s potmi povezan, saj točki $(0,0)$ in $(1, \sin 1)$ nista povezani
  s potjo. To je, ne obstaja zvezna preslikava $\gamma : [0,1] \to L$, za katero
  velja $\gamma(0) = (0,0)$ in $\gamma(1) = (1, \sin 1)$.
% lastnost 3
$L$ ni lokalno kompakten, vendar je zvezna slika lokalno kompaktnega
  prostora $\{-1\} \cup (0,1]$ za preslikavo
  %
  \begin{align*}
    f &: \{-1\} \cup (0,1] \to L \\
    f &: x \mapsto
    % 2. naloga:
    %   0         \text{če je } x = -1,
    %   \sinxinv  \text{če je } ??.
  \end{align*}
  %
% 
% lastnost 4
Topološka dimenzija $L$ je enaka~$1$.
% konec seznama


% 5. naloga:
Seveda na tem mestu ne moremo pojasniti vseh omenjenih topoloških pojmov.
Bralcu, ki se želi podučiti, priporočamo učbenika~!!,!!. 
Še mnogo drugih nenavadnih topološki prostorov pa najdemo v~!!.


\end{document}