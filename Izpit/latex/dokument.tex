\documentclass[10pt]{article}
\usepackage[a4paper, total={6in, 11.5in}]{geometry}
\usepackage[T1]{fontenc}
\usepackage[utf8]{inputenc}
\usepackage[slovene]{babel}
\usepackage{amsmath}
\usepackage{amsthm}
\usepackage{amssymb}
\usepackage{graphicx}
\title{Varšavski lok}
\author{Karol Borusok}
\date{20. januar 2026}
{\theoremstyle{definition}
\newtheorem{definicija}{Definicija}
}



\newcommand{\sinxinv}{\sin{\frac{1}{x}}}

\begin{document}
\thispagestyle{empty}

% 1. naloga:
% Varšavski lok
% Karol Borsuk
% 20. januar 2026
\maketitle
\begin{abstract}
Topologija je matematična veda, ki proučuje pojem \emph{prostor}. Študenti
matematike v prvem letniku spoznajo ekvlidske, vektorske in metrične prostore. A
topologi poznajo še veliko drugih prostorov, ki imajo nenavadne lastnosti. Eden takih je
tudi \emph{varšavski lok}, ki ga definiramo takole.
\end{abstract}
% 3. naloga: 
% začetek definicije
\begin{definicija}
  Varšavski lok $L$ je topološki podprostor ravnine, definiran s predpisom
  \begin{equation*}
    L \mathrel{{:}{=}} \left\{ \left(x, \sinxinv \right) \mid x \in (0,1]\} \cup \{(0,0) \right\}.
  \end{equation*}
\end{definicija}
% konec definicije

% 6. naloga: 
Na sliki~\ref{fig:sin-1-x} vidimo graf funkcije $y = \sinxinv$ na polodprtem intervalu $(0,1]$.
Če grafu dodamo še izhodišče~$(0,0)$, dobimo varšavki lok.
% 
\begin{figure}[ht]
  \centering
\includegraphics[width=0.5\textwidth]{warsaw.pdf}
  \label{fig:sin-1-x}
  \caption{Graf funkcije $y=\sinxinv$}
\end{figure}
%

Varšavski lok ima naslednje lastnosti:
% 
% 4. naloga: 
% 
% začetek seznama
\begin{enumerate}
% lastnost 1
\item $L$ je povezan prostor, kar pomeni, da ga ne moremo ga razdeliti na disjunktni odprti neprazni množici.
% 
% lastnost 2
\item $L$ ni s potmi povezan, saj točki $(0,0)$ in $(1, \sin 1)$ nista povezani
  s potjo. To je, ne obstaja zvezna preslikava $\gamma : [0,1] \to L$, za katero
  velja $\gamma(0) = (0,0)$ in $\gamma(1) = (1, \sin 1)$.
% lastnost 3
\item $L$ ni lokalno kompakten, vendar je zvezna slika lokalno kompaktnega
  prostora $\{-1\} \cup (0,1]$ za preslikavo
  %
  \begin{align*}
    f &: \{-1\} \cup (0,1] \to L \\
    f &: x \mapsto
  \begin{cases}
    0 & \text{če je } x = -1, \\
   \sinxinv & \text{če je } 0<x\leq1.
  \end{cases}
    % 2. naloga:
    %   0         \text{če je } x = -1,
    %   \sinxinv  \text{če je } ??.
  \end{align*}
  %
% 
% lastnost 4
\item Topološka dimenzija $L$ je enaka~$1$.
\end{enumerate}
% konec seznama


% 5. naloga:
\input{topologija.bib}
Seveda na tem mestu ne moremo pojasniti vseh omenjenih topoloških pojmov.
Bralcu, ki se želi podučiti, priporočamo učbenika~\ref{mrcun08},\ref{dugindji66}. 
Še mnogo drugih nenavadnih topološki prostorov pa najdemo v~\ref{steen78}.
\bibliographystyle{plain}
\end{document}